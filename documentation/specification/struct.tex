% 1. Structure of the language

\chapter{Overview of Scheme}

\section{Semantics}
\label{semanticsection}

\section{Syntax}

\section{Notation and terminology}

\subsection{Base and optional features}
\label{qualifiers}

\subsection{Error situations and unspecified behavior}
\label{errorsituations}

\subsection{Entry format}

% \noindent%
% \pproto{(make-vector \var{k})}{procedure}
% \pproto{(make-vector \var{k} \var{fill})}{procedure}\unpenalty


% \newcommand{\foo}[1]{\vr{#1}, \vri{#1}, $\ldots$ \vrj{#1}, $\ldots$}
% $$
% \begin{tabular}{ll}
% \vr{alist}&association list (list of pairs)\\
% \vr{boolean}&boolean value (\schtrue{} or \schfalse{})\\
% \vr{byte}&exact integer $0 \leq byte < 256$\\
% \vr{bytevector}&bytevector\\
% \vr{char}&character\\
% \vr{end}&exact non-negative integer\\
% \foo{k}&exact non-negative integer\\
% \vr{letter}&alphabetic character\\
% \foo{list}&list (see section~\ref{listsection})\\
% \foo{n}&integer\\
% \var{obj}&any object\\
% \vr{pair}&pair\\
% \vr{port}&port\\
% \vr{proc}&procedure\\
% \foo{q}&rational number\\
% \vr{start}&exact non-negative integer\\
% \vr{string}&string\\
% \vr{symbol}&symbol\\
% \vr{thunk}&zero-argument procedure\\
% \vr{vector}&vector\\
% \foo{x}&real number\\
% \foo{y}&real number\\
% \foo{z}&complex number\\
% \end{tabular}
% $$


% \begin{itemize}

% \item{It is an error if \var{start} is greater than \var{end}.}

% \item{It is an error if \var{end} is greater than the length of the
% string, vector, or bytevector.}

% \item{If \var{start} is omitted, it is assumed to be zero.}

% \item{If \var{end} is omitted, it assumed to be the length of the string,
% vector, or bytevector.}

% \item{The index \var{start} is always inclusive and the index \var{end} is always
% exclusive.  As an example, consider a string.  If
% \var{start} and \var{end} are the same, an empty
% substring is referred to, and if \var{start} is zero and \var{end} is
% the length of \var{string}, then the entire string is referred to.}

% \end{itemize}

\subsection{Evaluation examples}

% \begin{scheme}
% (* 5 8)      \ev  40%
% \end{scheme}

\subsection{Naming conventions}
