\chapter{Expressions}
\label{expressionchapter}

\newcommand{\syntax}{{\em Syntax: }}
\newcommand{\semantics}{{\em Semantics: }}

\section{Primitive expression types}
\label{primitivexps}

\subsection{Variable references}\unsection

\subsection{Literal expressions}\unsection
\label{literalsection}

\subsection{Procedure calls}\unsection

\subsection{Procedures}\unsection
\label{lamba}

\subsection{Conditionals}\unsection

\subsection{Assignments}\unsection
\label{assignment}

\begin{entry}{%
\proto{set!}{ \hyper{variable} \hyper{expression}}{\exprtype}}

\semantics
\hyper{Expression} is evaluated, and the resulting value is stored in
the location to which \hyper{variable} is bound.  It is an error if \hyper{variable} is not
bound either in some region\index{region} enclosing the {\cf set!}\ expression
or else globally.
The result of the {\cf set!} expression is
unspecified.

\begin{scheme}
(define x 2)
(+ x 1)                 \ev  3
(set! x 4)              \ev  \unspecified
(+ x 1)                 \ev  5%
\end{scheme}

\end{entry}

\subsection{Inclusion}\unsection
\label{inclusion}
\begin{entry}{%
\proto{include}{ \hyperi{string} \hyperii{string} \dotsfoo}{\exprtype}
\rproto{include-ci}{ \hyperi{string} \hyperii{string} \dotsfoo}{\exprtype}}

\semantics
Both \ide{include} and
\ide{include-ci} take one or more filenames expressed as string literals,
apply an implementation-specific algorithm to find corresponding files,
read the contents of the files in the specified order as if by repeated applications
of {\cf read},
and effectively replace the {\cf include} or {\cf include-ci} expression
with a {\cf begin} expression containing what was read from the files.
The difference between the two is that \ide{include-ci} reads each file
as if it began with the {\cf{}\#!fold-case} directive, while \ide{include}
does not.

\begin{note}
Implementations are encouraged to search for files in the directory
which contains the including file, and to provide a way for users to
specify other directories to search.
\end{note}

\end{entry}

\section{Derived expression types}
\label{derivedexps}

The constructs in this section are hygienic, as discussed in
section~\ref{macrosection}.
For reference purposes, section~\ref{derivedsection} gives syntax definitions
that will convert most of the constructs described in this section
into the primitive constructs described in the previous section.


\subsection{Conditionals}\unsection

\begin{entry}{%
\proto{cond}{ \hyperi{clause} \hyperii{clause} \dotsfoo}{\exprtype}
\pproto{else}{\auxiliarytype}
\pproto{=>}{\auxiliarytype}}

\syntax
\hyper{Clauses} take one of two forms, either
\begin{scheme}
(\hyper{test} \hyperi{expression} \dotsfoo)%
\end{scheme}
where \hyper{test} is any expression, or
\begin{scheme}
(\hyper{test} => \hyper{expression})%
\end{scheme}
The last \hyper{clause} can be
an ``else clause,'' which has the form
\begin{scheme}
(else \hyperi{expression} \hyperii{expression} \dotsfoo)\rm.%
\end{scheme}
\mainschindex{else}
\mainschindex{=>}

\semantics
A {\cf cond} expression is evaluated by evaluating the \hyper{test}
expressions of successive \hyper{clause}s in order until one of them
evaluates to a true value\index{true} (see
section~\ref{booleansection}).  When a \hyper{test} evaluates to a true
value, the remaining \hyper{expression}s in its \hyper{clause} are
evaluated in order, and the results of the last \hyper{expression} in the
\hyper{clause} are returned as the results of the entire {\cf cond}
expression.

If the selected \hyper{clause} contains only the
\hyper{test} and no \hyper{expression}s, then the value of the
\hyper{test} is returned as the result.  If the selected \hyper{clause} uses the
\ide{=>} alternate form, then the \hyper{expression} is evaluated.
It is an error if its value is not a procedure that accepts one argument.  This procedure is then
called on the value of the \hyper{test} and the values returned by this
procedure are returned by the {\cf cond} expression.

If all \hyper{test}s evaluate
to \schfalse{}, and there is no else clause, then the result of
the conditional expression is unspecified; if there is an else
clause, then its \hyper{expression}s are evaluated in order, and the values of
the last one are returned.

\begin{scheme}
(cond ((> 3 2) 'greater)
      ((< 3 2) 'less))         \ev  greater%

(cond ((> 3 3) 'greater)
      ((< 3 3) 'less)
      (else 'equal))            \ev  equal%

(cond ((assv 'b '((a 1) (b 2))) => cadr)
      (else \schfalse{}))         \ev  2%
\end{scheme}


\end{entry}
