\clearextrapart{Introduction}
\label{historysection}

CX is both a compiled and interpreted, strongly typed programming
language. The distinctive features of CX are
affordances, which tell us what actions can be done on an element of a
program; program serialization, which can be performed even to a running
program; and program stepping, which allows the programmer to control
a program's execution to go backward or forward a desired number of steps.

In order to provide the core features previously mentioned, a CX
implementation is required to reach a very flexible program
structure. CX programs can have access to their own internal structure during
compilation and runtime; they can control their own execution, so, in
theory, they can step back, modify their structure, and command
themselves to continue. In a distributed system, a CX program could
fully or partially serialize itself, send its serialized self to another
server, and resume its execution there.

Affordances in CX represent a finite set of actions that can be
performed over a CX program element, e.g., adding arguments to a
function call or adding an expression to a function definition. There are
primitive rules which determine what actions are allowed or
restricted, such as type restrictions (an operator can only receive
arguments of certain types) and what are the currently defined
functions in the program. CX provides a more complex mechanism for
determining an element's affordances, which is based on Prolog clauses
and queries. Using facts and rules, a programmer can describe the environment in
which a module is compiled or run, and the module will change its behaviour
accordingly.

% A CX program can be serialized to byte arrays, and deserialized back
% at any moment. The call stack of a CX program is also serialized,
% which enables a deserialized program to continue its execution.

% A CX program can be instructed to run only for a given number of steps
% or calls. Also, a CX This program stepping allows the programmer to have more control

The CX specification describes a set of constructs (called ``base
language''), which can then be used by other programs (particularly,
parsers) to generate CX programs. As a consequence, any programming
language can be regarded as a CX dialect as long as they internally follow
these base constructs. The present specification file also includes
the description of an implementation of a CX base language and 
lexical and syntactic analyzers for a programming language that
generates CX base code. This CX implementation is programmed in Go.





% \subsection*{Background}

% \subsection*{Acknowledgments}

