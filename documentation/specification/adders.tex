% Lexical structure

%%\vfill\eject
\chapter{Adders}
\label{adders}

In CX, adders are the basic building tools to construct programs. A CX
program begins as an empty object, and new elements can be attached to
it by the use of an adder. There exists one adder for each type of
element that can be attached to a program. An adder will take the
element to be attached, and will perform the binding to a parent element. In this binding
process, the element can be pre-processed before

The adder can pre-process the element before this binding process,
e.g., the expression adder can also attach the module to whom it will
belong to (which should be its parent function's module). For this
reason, it is convenient to have a series of helper functions solely
for the task of creating an element, and restricting the adders to
only perform the binding process. In this specification file,
\textit{makers} (see Section \ref{makers}) are suggested for element
creation. However, it is important to note that adders could be in
charge of both creating the element and performing the binding
process.

An adder must be used whenever a CX program needs to change its
element structure. For example, one could be tempted to directly
append an element to another; considering a CX implementation which
follows a Go-like syntax:

\begin{lstlisting}
  program.Modules = []Module{mod1, mod2}
\end{lstlisting}

The above code could raise multiple problems. One of these problems is
that it could potentially lose all the previously appended
modules. Another problem is that the modules being added don't have a
reference to the main program structure.

The output value of an adder is not specified.

The following sections describe the adders for each of the elements
that can be present in a CX program.

\section{AddModule}
\label{addmodule}

\textit{AddModule} takes a CX element with at least a name which
uniquely identifies it among the other modules. This means that a bare
representation of a module in CX is a name. This adder should attach
to the module a reference to the program structure

\begin{lstlisting}
  mod.Program = &program
\end{lstlisting}

as in the example above. \textit{AddModule} must also inform the
program struct that the current module (see Section \ref{structs}) will be the module just
added. An example of this process is:

\begin{lstlisting}
  prgrm.CurrentModule = &mod
\end{lstlisting}

As mentioned above, a module must have a name that uniquely identifies
it. This implies that there can not be two modules with the same name,
and \textit{AddModule} should be in charge of avoid duplicated
modules. It is not specified if \textit{AddModule} should or should
not redefine a module if a duplicated module is sent as an argument.

\section{AddDefinition}
\label{adddefinition}

\textit{AddDefinition} takes a CX element with at least a name, an
array of bytes representing the definition's value, and a type. The
name is used to uniquely identify the definition among other
definitions. The type is used for instructing a compiler or an
interpreter how should the array of bytes be read. This adder should
attach to the definition a reference to the program structure, as well
as a reference to the module which is going to contain it. An example
of this process is:

\begin{lstlisting}
  def.Program = &prgrm
  def.Module = &mod
\end{lstlisting}

A definition must have a name that uniquely identifies it, which
implies that there can not be two definitions with the same
name. \textit{AddDefinition} should be in charge of avoiding
duplicated definitions. If a definition with an already defined name is sent as
an argument to \textit{AddDefinition}, the default behaviour must be to redefine
the definition. If this was not the case, every definition would be
treated as constants in a CX program. It is a task for the parser and
code generator to differentiate between a constant and a variable
definition.

\section{AddFunction}
\label{addfunction}

\textit{AddFunction} takes a CX element with at least a name, which is used to
uniquely identify the function among other functions defined in a module. This
adder should attach to the function a reference to the program structure, as
well as a reference to the module which is going to contain it

\begin{lstlisting}
  fn.Program = &prgrm
  fn.Module = &mod
\end{lstlisting}

as in the example above. \textit{AddFunction} should also set the module's
current function (see Section \ref{structs}) to the one which is being added. An
example of this process is:

\begin{lstlisting}
  prgrm.CurrentFunction = &mod
\end{lstlisting}

A function must have a name that uniquely identifies it, which implies that
there can not be two functions with the same name. \textit{AddFunction} should
be in charge of avoiding duplicated functions. It is not specified whether a
function with a duplicated name should or should not redefine the old instance.

\section{AddStruct}
\label{addstruct}

\textit{AddStruct} takes a CX element with at least a name, which is used to
uniquely identify the struct among other structs defined in a module. This adder
should attach to the struct a reference to the program structure, as well as a
reference to the module which is going to contain it

\begin{lstlisting}
  strct.Program = &prgrm
  strct.Module = &mod
\end{lstlisting}

as in the example above. \textit{AddStruct} should also set the module's current
structure (see Section \ref{structs}) to the one which is being added. An
example of this process is:

\begin{lstlisting}
  strct.CurrentStrct = &strct
\end{lstlisting}

\section{AddImport}
\label{addimport}

An import element in CX is a synonym for a CX module element. As a consequence,
\textit{AddImport} must accept the same type and number of arguments as
\textit{AddModule}. The difference between these two adders is in the process:
\textit{AddModule} attaches a new module to a program, which can hold
definitions, functions, structs, etc., while \textit{AddImport}  attaches a
module reference to another module. Once a module is attached as an import to
another module, elements in the module have access to the elements contained in
the imported module (see Section \ref{overview}).

\textit{Addimport} should be in charge of avoiding duplicated imported
modules. As redefining an import in a module should not add any behaviour to a
CX program, a duplicated import being sent to \textit{AddImport} should be
discarded.

\section{AddObject}
\label{addobject}

Prolog objects (see Section \ref{overview} and Section \ref{structs}) are
contained in CX module elements. As objects are stored as just a collection of
names, the only task of \textit{AddObject} should be to append a new name to
this collection. Despite the process being so simple, it should be encapsulated
for maintainability purposes.

\section{AddClauses}
\label{addclauses}

Prolog clauses (see Section \ref{overview} and Section \ref{structs}) are
contained in CX module elements. As clauses are stored as just a string of
characters, the only task of \textit{AddClauses} should be to redefine this
property of a CX module element. Despite the process being so simple, it should
be encapsulated for maintainability purposes.

\section{AddQuery}
\label{addquery}

Prolog queries (see Section \ref{overview} and Section \ref{structs}) are
contained in CX module elements. As a query is stored as just a string of
characters, the only task of \textit{AddClauses} should be to redefine this
property of a CX module element. Despite the process being so simple, it should
be encapsulated for maintainability purposes.

\section{AddField}
\label{addfield}

As is explained in Section \ref{structs}, a CX field element consists of only a
name and a type, the process of adding a field to a CX struct element should be
simple. \textit{AddField} should be in charge of avoiding duplicate fields
in a CX struct element and perform the actual appending of the field to the
struct. If the field name is not present in the fields collection, it should be
appended.

It is not specified whether a field should be redefined or not if the field name
to be appended is already present in the fields collection.

\section{AddExpression}
\label{addexpression}

\textit{AddExpression} takes a CX element with at least a CX function element. This
adder should attach to the expression a reference to the program structure, a
reference to the module which is going to contain it, as well as a reference to
the function which is going to call it during program execution. An example of
attachments to these references is shown below

\begin{lstlisting}
  expr.Program = &prgrm
  expr.Module = &mod
  expr.Function = &fn
\end{lstlisting}

\textit{AddExpression} should also set the function's
current expression (see Section \ref{structs}) to the one which is being
added. An example of this process is:

\begin{lstlisting}
  fn.CurrentExpression = &expr
\end{lstlisting}

Lastly, \textit{AddExpression} should also be in charge of assigning a line
number (see Section \ref{structs}) to the expression being added. This line
number should be equal to number of expressions contained by the function before
the current expression is added.

\section{AddInput}
\label{addinput}

As is explained in Section \ref{structs}, a function's input is defined by a CX
parameter element, and this element consists of only a name and a type. As a
consequence, the process of adding an input to a CX function element should be
simple: append the new parameter to the list of inputs attached to the
function, and avoid duplicated input parameter elements by name. 

If a parameter to be appended is already present in the inputs collection, this
parameter must be discarded.

\section{AddOutput}
\label{addoutput}

As is explained in Section \ref{structs}, a function's output is defined by a CX
parameter element, and this element consists of only a name and a type. As a
consequence, the process of adding an output to a CX function element should be
simple: append the new parameter to the list of outputs attached to the
function, and avoid duplicated output parameter elements by name. 

If a parameter to be appended is already present in the outputs collection, this
parameter must be discarded.

As explained in Section \ref{overview}, a feature of CX is its capability of
defining functions with multiple outputs. This behaviour implies that one can
call \textit{AddOutput} multiple times to append new outputs, and the output
would not be redefined.

\section{AddArgument}
\label{addargument}

\textit{AddArgument} takes a CX element with at least a byte array which defines
its value, and a type which instructs the compiler or interpreter how should the
array of bytes be read. The CX argument element is appended to an ordered
collection of arguments attached to an expression.

Any argument can be attached to any expression, and no constraints or
limitations exist, as it's the job of the compiler or interpreter to raise an
error in case there exists an input number or type mismatch.

\section{AddOutputName}
\label{addoutputname}

\textit{AddOutputName} takes a character string which is used to construct a CX
definition. The constructed definition is appended to a collection of output
names attached to a CX expression element. \textit{AddOutputName} must take
information from the expression's operator in order to determine the
definition's type. The definition's value must be initialized to a
zero-equivalent value (e.g., 0 for an i32 type, an empty string for a str type).

