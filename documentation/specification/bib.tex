
% My reference for proper reference format is:
%    Mary-Claire van Leunen.
%    {\em A Handbook for Scholars.}
%    Knopf, 1978.
% I think the references list would look better in ``open'' format,
% i.e. with the three blocks for each entry appearing on separate
% lines.  I used the compressed format for SIGPLAN in the interest of
% space.  In open format, when a block runs over one line,
% continuation lines should be indented; this could probably be done
% using some flavor of latex list environment.  Maybe the right thing
% to do in the long run would be to convert to Bibtex, which probably
% does the right thing, since it was implemented by one of van
% Leunen's colleagues at DEC SRC.
%  -- Jonathan

% I tried to follow Jonathan's format, insofar as I understood it.
% I tried to order entries lexicographically by authors (with singly
% authored papers first), then by date.
% In some cases I replaced a technical report or conference paper
% by a subsequent journal article, but I think there are several
% more such replacements that ought to be made.
%  -- Will, 1991.

% This is just a personal remark on your question on the RRRS:
% The language CUCH (Curry-Church) was implemented by 1964 and 
% is a practical version of the lambda-calculus (call-by-name).
% One reference you may find in Formal Language Description Languages
% for Computer Programming T.~B.~Steele, 1965 (or so).
%  -- Matthias Felleisen

% Rather than try to keep the bibliography up-to-date, which is hopeless
% given the time between updates, I replaced the bulk of the references
% with a pointer to the Scheme Repository.  Ozan Yigit's bibliography in
% the repository is a superset of the R4RS one.
% The bibliography now contains only items referenced within the report.
%  -- Richard, 1996.

% Once again, the bibliography now contains only items referenced within the report.
%  -- John Cowan, 2013

\begin{thebibliography}{999}

\bibitem{SICP}
Harold Abelson and Gerald Jay Sussman with Julie Sussman.
{\em Structure and Interpretation of Computer Programs, second edition.}
MIT Press, Cambridge, 1996.

\bibitem{Bawden88}
Alan Bawden and Jonathan Rees.
Syntactic closures.
In {\em Proceedings of the 1988 ACM Symposium on Lisp and
  Functional Programming}, pages 86--95.

\bibitem{rfc2119}
S. Bradner.
Key words for use in RFCs to Indicate Requirement Levels.
\url{http://www.ietf.org/rfc/rfc2119.txt}, 1997.

\end{thebibliography}
